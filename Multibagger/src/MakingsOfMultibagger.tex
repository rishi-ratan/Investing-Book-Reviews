\documentclass[english,titlepage]{article}

\usepackage[pdftex]{graphicx}
\usepackage[tmargin=0.8in, bmargin=0.8in, rmargin=0.8in, lmargin=0.8in]{geometry}
\usepackage{setspace}
\usepackage{booktabs}
\usepackage{bigstrut}
\usepackage{subfigure}
\usepackage{float}
\usepackage{hyperref}
\usepackage{longtable}
\usepackage{rotating}
\usepackage[inline]{enumitem}
\usepackage{xspace}
%\usepackage{fullpage}
\usepackage{svg}
\usepackage{amsmath}
\usepackage{comment}
\usepackage{amssymb}
\usepackage{amsfonts}
\usepackage{multicol}
\usepackage{listings}
\usepackage{float} % use for putting pics where you want. Replace the ht with H
\usepackage{verbatim}
\usepackage{placeins}
\usepackage{caption}
\usepackage[most]{tcolorbox}
\setlength{\parindent}{0cm} % removes paragraph indentation 

\makeatletter
\usepackage{fancyhdr}
\fancyhead{}
\fancyhead[L]{\bfseries Michael Mauboussin}
\fancyhead[C]{\bfseries Expectations Investing}
\fancyhead[R]{\bfseries Columbia Business School}
\fancyfoot[C]{\thepage}
\fancyfoot[L]{Rishi Ratan}
\fancyfoot[R]{Jan 2022}
\pagestyle{fancy}
\renewcommand{\headrulewidth}{0.5pt}
\renewcommand{\footrulewidth}{0.5pt}
\makeatother
\usepackage{babel}
\begin{document}
\title{\textbf{Expectations Investing} \\ 
\date{}
\doublespacing{Key Learnings Summary}}
\maketitle
\section{Key Shifts in the market since 2001}
\begin{itemize}
    \item \textbf{Shift from \emph{active to passive} investing}: Index funds and ETFs are the prominent investment asset classes rather than stock picking. 
    \item \textbf{Rise of intangible investments}: Companies today invest substantially more in intangible assets because they appear on the income statement as an expense, while tangible investments are recorded as assets on the balance sheet. 
    \item \textbf{Shift from Public to Private Equity}: There are ~1/3 fewer public companies listed in the US today compared to 2001. VC an PE have become the new preferred asset classes for investors and operators are staying private longer to avoid scrutiny endured in the public markets. 
    \item \textbf{Changes in Accounting Rules}: In the 1990s, stock-based compensation (SBC) consisted primarily of employee stock options (ESOPs) that were \emph{not} expensed on the income statement. 
                                                Today, SBC is primarily in the form of restricted stock units (RSUs) that are expensed on the income statement. 
                                                Therefore, both the form of renumeration and how it is accounted for have changed. Furthermore, the accounting rules for M\&A have revised \emph{ending} the pooling-of-interests method and eliminating goodwill amortization. 
\end{itemize}
\section{Chapter 1: Case for Expectations Investing}
Stock prices are the \textbf{clearest} and most \textbf{reliable} signal of the market's expectations about a company's future financial performance. 
\begin{tcolorbox}[colback=blue!5!white,colframe=blue!75!black]
    The key to successful investing is to estimate the level of expected performance embedded in the current stock price and then assess the likelihood of a revision in expectations. 
\end{tcolorbox}

\end{document}